\documentclass{article}
\usepackage[utf8]{inputenc}
\usepackage{graphicx}
\usepackage[margin=0.5in]{geometry}
\usepackage{pgfplots}
\usepackage{longtable}
\usepackage[T1]{fontenc}
\usepackage[polish]{babel}
\usepackage[export]{adjustbox}
\usepackage{hyperref}
\title{First project}
\author{Michał Głuszkiewicz}
\date{December 2022}
\pgfplotsset{width=10cm,compat=1.9}
\begin{document}

\maketitle
\section{Wprowadzenie}
\begin{center}
Wolny spadek jest to ruch jednostajnie przyspieszony,którego przyspieszenie wynosi: $g = 9,8m/s^2$
\end{center}
\begin{equation}
h(t)=h_0 - \frac{gt^2}{2}
\end{equation}
W spadku swobodnym czas spadania ts ciała upuszczonego z wysokości h możemy obliczyć ze wzoru:
\begin{equation}
    t_s= \sqrt{\frac{2h}{g}}
\end{equation}
\section{Przebieg eksperymentu}
Eksperyment wygląda w taki sposób, że upuszczamy ciało z jakiejś wysokości i liczymy w jakim czasie pokonuje kolejne odległości w spadku swobodnym. Trzeba jednak wiedzieć, że droga w realnym życiu nigdy nie jest idealna i może być odchylenie/niepewność która jest zaznaczana na wykresie.

\begin{figure}[h]
\includegraphics[scale=0.6,center]{rys0010.jpg}
\label{fig:schemat doswiadczenia}
\caption{spadek swobodny}
\end{figure}
\section{Wyniki pomiaru}
\begin{center}
\begin{tikzpicture}
\begin{axis}[
    axis lines = left,
    xlabel = \(T(s)\),
    ylabel = {\(S(m)\)},
]
\addplot [
    mesh,
    samples=100, 
    domain = 0:10,
    color=red,
]
{x^2 * 9.8 / 2};
\addplot+[
    only marks,
    mark size=0.5pt,
    color=blue,
    each nth point={2}]
table[meta=s]
{pomoclatex.csv};
\addlegendentry{\(x = gt^2/2\)}
\addlegendentry{\(dots = s+norma\)}
\end{axis}
\end{tikzpicture}
\end{center}
\begin{longtable}[c]{|c|c|c|c|}
\caption{dane do wykresu\label{long}}\\
 \hline
 \multicolumn{4}{| c |}{Poczatek tabeli}\\
 \hline
 t[s] & s[m] & norma & s+norma\\
 \hline
 \endfirsthead

 \hline
 \multicolumn{4}{|c|}{Kontynuacja tabeli \ref{long}}\\
 \hline
 t[s] & s[m] & norma & s+norma\\
 \hline
 \endhead

 \hline
 \endfoot

 \hline
 \multicolumn{4}{| c |}{Koniec tabeli}\\
 \hline
 \endlastfoot

0 & 0,00 & -0,54 & -0,54\\
0,1 & 0,05 & -0,09 & -0,04\\
0,2 & 0,20 & -2,68 & -2,49\\
0,3 & 0,44 & 1,89 & 2,33\\
0,4 & 0,78 & 1,31 & 2,09\\
0,5 & 1,23 & -1,33 & -0,11\\
0,6 & 1,76 & 0,56 & 2,32\\
0,7 & 2,40 & -1,45 & 0,95\\
0,8 & 3,14 & 1,03 & 4,16\\
0,9 & 3,97 & -3,13 & 0,83\\
1 & 4,90 & 0,79 & 5,69\\
1,1 & 5,93 & 0,12 & 6,05\\
1,2 & 7,06 & 1,66 & 8,71\\
1,3 & 8,28 & 5,09 & 13,37\\
1,4 & 9,60 & -0,50 & 9,10\\
1,5 & 11,03 & -2,88 & 8,15\\
1,6 & 12,54 & 1,79 & 14,33\\
1,7 & 14,16 & -0,33 & 13,83\\
1,8 & 15,88 & 1,54 & 17,42\\
1,9 & 17,69 & 0,41 & 18,10\\
2 & 19,60 & -2,24 & 17,36\\
2,1 & 21,61 & 0,98 & 22,59\\
2,2 & 23,72 & 0,72 & 24,44\\
2,3 & 25,92 & -1,24 & 24,68\\
2,4 & 28,22 & 1,11 & 29,33\\
2,5 & 30,63 & -2,30 & 28,33\\
2,6 & 33,12 & 1,02 & 34,15\\
2,7 & 35,72 & 0,71 & 36,43\\
2,8 & 38,42 & 0,63 & 39,04\\
2,9 & 41,21 & -0,19 & 41,02\\
3 & 44,10 & 4,57 & 48,67\\
3,1 & 47,09 & -1,38 & 45,71\\
3,2 & 50,18 & -1,89 & 48,29\\
3,3 & 53,36 & 0,36 & 53,72\\
3,4 & 56,64 & 0,95 & 57,59\\
3,5 & 60,03 & -2,10 & 57,93\\
3,6 & 63,50 & -1,81 & 61,69\\
3,7 & 67,08 & -0,40 & 66,69\\
3,8 & 70,76 & 1,95 & 72,71\\
3,9 & 74,53 & -0,98 & 73,55\\
4 & 78,40 & -1,87 & 76,53\\
4,1 & 82,37 & 3,16 & 85,53\\
4,2 & 86,44 & -0,62 & 85,81\\
4,3 & 90,60 & 1,56 & 92,16\\
4,4 & 94,86 & -1,01 & 93,85\\
4,5 & 99,23 & -0,22 & 99,01\\
4,6 & 103,68 & 1,18 & 104,86\\
4,7 & 108,24 & 1,22 & 109,46\\
4,8 & 112,90 & 2,17 & 115,06\\
4,9 & 117,65 & -0,11 & 117,54\\
5 & 122,50 & 0,47 & 122,97\\
5,1 & 127,45 & 0,82 & 128,27\\
5,2 & 132,50 & 0,81 & 133,30\\
5,3 & 137,64 & -2,25 & 135,39\\
5,4 & 142,88 & 2,47 & 145,35\\
5,5 & 148,23 & 1,60 & 149,83\\
5,6 & 153,66 & -1,17 & 152,49\\
5,7 & 159,20 & 0,97 & 160,17\\
5,8 & 164,84 & -0,57 & 164,27\\
5,9 & 170,57 & 0,15 & 170,72\\
6 & 176,40 & 3,04 & 179,44\\
6,1 & 182,33 & -0,23 & 182,10\\
6,2 & 188,36 & 0,17 & 188,52\\
6,3 & 194,48 & 1,65 & 196,13\\
6,4 & 200,70 & 0,05 & 200,75\\
6,5 & 207,03 & -0,86 & 206,17\\
6,6 & 213,44 & -1,32 & 212,12\\
6,7 & 219,96 & 1,53 & 221,50\\
6,8 & 226,58 & 1,92 & 228,49\\
6,9 & 233,29 & -0,99 & 232,30\\
7 & 240,10 & -0,52 & 239,58\\
7,1 & 247,01 & 2,68 & 249,69\\
7,2 & 254,02 & -0,25 & 253,76\\
7,3 & 261,12 & -1,51 & 259,61\\
7,4 & 268,32 & 0,97 & 269,30\\
7,5 & 275,63 & 0,54 & 276,17\\
7,6 & 283,02 & 0,52 & 283,54\\
7,7 & 290,52 & 1,99 & 292,51\\
7,8 & 298,12 & 2,03 & 300,15\\
7,9 & 305,81 & -1,60 & 304,21\\
8 & 313,60 & 0,48 & 314,08\\
8,1 & 321,49 & 2,41 & 323,89\\
8,2 & 329,48 & -3,19 & 326,29\\
8,3 & 337,56 & -0,86 & 336,70\\
8,4 & 345,74 & 0,13 & 345,88\\
8,5 & 354,03 & 1,82 & 355,85\\
8,6 & 362,40 & 0,92 & 363,33\\
8,7 & 370,88 & 1,03 & 371,91\\
8,8 & 379,46 & 0,19 & 379,64\\
8,9 & 388,13 & -0,74 & 387,39\\
9 & 396,90 & 0,97 & 397,87\\
9,1 & 405,77 & -0,84 & 404,92\\
9,2 & 414,74 & 4,22 & 418,96\\
9,3 & 423,80 & -1,86 & 421,94\\
9,4 & 432,96 & 0,65 & 433,62\\
9,5 & 442,23 & -1,58 & 440,64\\
9,6 & 451,58 & -1,43 & 450,16\\
9,7 & 461,04 & 0,97 & 462,01\\
9,8 & 470,60 & -1,31 & 469,28\\
9,9 & 480,25 & 3,10 & 483,35\\
10 & 490,00 & 0,09 & 490,09\\
 \end{longtable}
\section{Wnioski}
\begin{figure}[h]
\includegraphics[width=0.6\textwidth, center]{wykres1.png}
\label{fig:wyniki pomiaru 1}
\caption{histogram bledow pomiaru}
\end{figure}
\begin{itemize}
\item Brak zależności masy

\item Czas spadku różnych obiektów taki sam

\item Poruszanie po linii prostej w dół(~Newton)

\item Spadek rozpoczyna się od spoczynku(V=0m/s)
\end{itemize}
\newpage
\tableofcontents

\end{document}